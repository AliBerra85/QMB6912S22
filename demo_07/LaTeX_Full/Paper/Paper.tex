
%%%%%%%%%%%%%%%%%%%%%%%%%%%%%%%%%%%%%%%%
\documentclass{paper}
%%%%%%%%%%%%%%%%%%%%%%%%%%%%%%%%%%%%%%%%

% Declare packages in the preamble. 

% Declare some TeX packages needed for graphics. 
\usepackage{graphicx}
\usepackage{epstopdf}
\epstopdfsetup{suffix=}

% Sometimes it helps to use an if statement, 
% if you use the file on multiple platforms. 
% \ifx\pdftexversion\undefined
%     \usepackage[dvips]{graphicx}
% \else
%     \usepackage[pdftex]{graphicx}
%     \usepackage{epstopdf}
%     \epstopdfsetup{suffix=}
% \fi


%%%%%%%%%%%%%%%%%%%%%%%%%%%%%%%%%%%%%%%%
\begin{document}
%%%%%%%%%%%%%%%%%%%%%%%%%%%%%%%%%%%%%%%%

\section{Sample Document}

This is a sample of a document generated automatically from the figures and tables produced by
the script that was used to read in and analyze data.
First, in Section \ref{sec:data}, it describes the data.
Next, in Section \ref{sec:results}, it describes the results of the estimated regression model.

\section{Data} \label{sec:data}

Summary statistics for numerical variables are shown in Table \ref{tab:summary} (all figures in millions).

\input{../Tables/summary.tex}


Table \ref{tab:earthquakes} shows the frequency of observations in and out of California along with the incidence of earthquakes. Notice that earthquakes have only happened in California.

\input{../Tables/earthquakes.tex}

The correlation matrix of potential variables in the model is shown in Table \ref{tab:corr}.
House prices are positively correlated with income and California but negatively correlated with earthquakes. In the next setion, these variables will be included in a regression model.

% latex table generated in R 4.1.1 by xtable 1.8-4 package
% Thu Feb 17 20:55:06 2022
\begin{table}[ht]
\centering
\begin{tabular}{rrrrr}
  \hline
 & Log. of Price & Horsepower & Age & Engine Hours \\ 
  \hline
Log. of Price & 1.000 & 0.649 & -0.441 & -0.046 \\ 
  Horsepower & 0.649 & 1.000 & 0.039 & 0.378 \\ 
  Age & -0.441 & 0.039 & 1.000 & 0.559 \\ 
  Engine Hours & -0.046 & 0.378 & 0.559 & 1.000 \\ 
   \hline
\end{tabular}
\caption{Correlation Matrix of Numeric Variables} 
\label{tab:correlation}
\end{table}




\pagebreak
\section{Empirical Results}  \label{sec:results}


The estimates from the regression model are shown in Table \ref{tab:lm_model_1}.
\input{../Tables/lm_model_1.tex}

\input{../Text/regression.tex}




\pagebreak
The regression lines are shown in Figure \ref{fig:reg}, with the estimated intercept term for zip codes affected by earthquakes (red), those in the rest of California (green), and the zip codes outside of California.

\begin{figure}
\centering
\includegraphics[width=\textwidth]{../Figures/regression.eps}
\caption{Regression Model Predictions}
\label{fig:reg}
\end{figure}


\pagebreak
The predictions are shown in Figure \ref{fig:pred} compared to observed prices.
It is clear that there is a reasonably close relationship between the predictions and the observed prices.

\begin{figure}
\centering
\includegraphics[width=\textwidth]{../Figures/predictions.eps}
\caption{House Prices vs. Predicted Prices}
\label{fig:pred}
\end{figure}




%%%%%%%%%%%%%%%%%%%%%%%%%%%%%%%%%%%%%%%%
\end{document}
%%%%%%%%%%%%%%%%%%%%%%%%%%%%%%%%%%%%%%%%
